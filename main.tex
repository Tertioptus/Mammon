\documentClass[11pt]{book}
\title{\bf Mammon}
\author{Benjamin Paige III}
\date{\today}

\begin{document}

Locked-In

Time Management:
Mode Map
Task Queue

DDM (Dynamic Defined Modality)

Then why doesn't everybody do it?
	The next question is that if this type of planning is so crucial and so easy, why doesn't everybody do it.  We it's easy in comparison building a pyramid out of dirt and hay.
	Yet it can be a time consuming venture, that delays the availing of any tangible product. Much like gardening, in that most of the work is front loaded, and the harvest comes
	after an extended amount of time.  Plainly, it requires a great deal of faith.  Whereas you can decide to just eat grass.  And unless your a cow, you become sick.  
	Most people are either cows or vocationally sick.  The rest of us cultivate our time -- so that it will work for us.

	For instance, there two very effective ways to study from a textbook: hunker down and read through the entire book in one setting, or have a sophisticated plan for studying the
	entire books content iterably, without losing context.  The former does not allow for many other activites, like the ones that will allow you to not die during education.  While
	the latter seems daunting, because it entails extra work on top of an already consirably heavy load.  Obviously, reading the entire book is out of the question.  So we must 
	figure out how to make study planning palatable.

	What makes planning for study difficult isn't just literal action of note taking and keeping, it is everything that comes along with it.  In order to maintain a general context
	of the source material, one would have to not just keep well-structured, indexed and contextual notes, but a routine review ordinance and curation processes must be enacted as well.
	Basically, you will need to routinely take a portion of time for study and note taking; And for each note taken, you will need to store it such a way that in the future you will be able to find it quickly, and understand it's context.

	Which means you will need to be organized with your daily schedule and in your environment.  To accomplish all this, just to study a book, it what I like to call work.  And it is neccessary, but it's not always fun; Yet, 
	when you break the work down, it can be made palatable, and even addictive (in a good way).

words not used:
predilection


concepts not broached:

\begin{itemize}
\item
"increase vocabulary by rewriting personal documents, and/or adding new terms to a contextual reference sheet.
.e. a finance reference sheet would contain terms like escrow, etc"

The abacus theory: how kids in china are taught to work an abacus with extreme efficiency
to solve complicated math problems very fast, eventually they don't even need the abacus
I believe there is a corrolationi here with defined dynamic modality
\end{itemize}
\end{document}
